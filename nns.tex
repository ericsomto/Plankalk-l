
$Free Choice of Denotation
For all objects, freely chosen identifiers (Bezeichnun-
gen) may be introduced; for example, a standard deno-
tation can be associated (zugeordnet) with an identifier
(see Section 6) such that both possess the same object
as their value (Wert).
In a Rechenplan 6', i.e. in a program or a subroutine
(see Section 9), an identifier is a letter followed by a
number. The letter is V, Z, R, or C, depending on
whether the object in question is used as an input
parameter (Variable), intermediate value (Zwischen-
wert), result parameter (Resultatwert), or as a constant
in 6~. The distinguishing number (Nummer) is attached
to the letter in the line below. The letter classifies the
objects.
Examples:
II, Z, Z, R
0 0 1 0
Finally, programs and subroutines have their own
identifiers like
P12, P3.7
the number following the letter P being a program
index (Programm-lndex), in the form of a component-
subscript (see Section 4). The second example denotes
"the program 7 of the program group 3." Thus, Zuse
has arrays of programs and a corresponding block struc-
ture. He derives from this a system to denote the results
of subroutines in external use; for example, the result
R of a subroutine P17 is external to P17 characterized
0
by the program index 17, i.e. by
R17
0
which also involves a call of P17 

. Subseripting
The selection of a component is achieved with the
help of a component-subscript (Komponenten-Index),
that is the denotation of a number (simple subscript)
or a sequence of numbers (multiple subscript). The
component-subscript is written immediately under the
identifying number of the corresponding composite ob-
ject.
V denote an array of the modeLet, for example, 0
I X m X S l . n , then
V
0 ( 0 < i < / )
i
selects its ith component, a subarray of a structure
m X S l . n , while
V
0
i'j
( 0 _ < j < m)
V
selects the jth component of 0 , a list of the structure
i
S l . n , and finally
V
0 ( 0 < k < n )
i . j . k
V
selects the kth component of 0 , a single bit. In today's
i.j
notation, this corresponds to V0[i], VO[i,j], VO[i,j, k]

The form o f denotation with a " m a i n line" and
"index lines" V and K for variable-number and c o m -
ponent-subscript, respectively, is supplemented by an
optional c o m m e n t line S, in which the structure or m o d e
o f the value in question can be noted. To this end, the
n o t a t i o n o f Section 1 is used; Zuse calls these indications
Struktur-lndizes.
Examples are given in Figure 2, an illustrative sec-
tion f r o m [Z59, p. 69].
The explicit m a r k i n g o f the lines by prefixed letters
V, K, and S, allows one to omit e m p t y K-lines. F u r -
thermore, the prefix S in the m o d e denotation can be
dropped. Thus,
S[ S l . n m >( S l . n SO $2 ~r
can be shortened to
S[ 1.n m X 1.n 0 2 a.
Moreover, Zuse allows the abbreviation o f
S[ A1 A2 

A [ 1 2 0 3
(using A 0 s y n o n y m ously with S0)
F u r t h e r m o r e , variable c o m p o n e n t subscripts can be
used [Z70, p. 123], for example by the help o f an inter-
mediate value in the f o r m
K V ZV 0 I
k
rn X 1.n 1.n
with the m e a n i n g o f V0[Z1] in t o d a y ' s notation.
( N o t e that Z1 is o f structure S1 .n; that is, the integer
corresponding to the bit sequence Z I is used as sub-
script, and a c o m p o n e n t o f structure S 1 - n is selected
f r o m VO.)
Originally, Zuse [Z49] introduced the ~ , shaped equality
sign. The arrow-like sign ==;,is used in IZ59], after Rutishauser
had helped to propagate it. In [R52], Rutishauser used in typesclipt
the sign#=. At the Zfirich ALGOL Conference 1958, the sign
:= was introduced under strong pressure from the American
participants. The European group wished to use Zuse's sign.
681

Assignment and Identity Declaration
The m o s t i m p o r t a n t feature for the construction o f
p r o g r a m s is the assignment (Rechenplangleichung), ex-
pressed by means o f the Ergibtzeichen ~ . 9 F o r ex-
ample, the assignment
1Zn + 1 ~ ZV 1
S 1 .n 1 .n
means to a u g m e n t the integer intermediate value Z1 by
1, while
(z, v ) ~ R
V 0 1 0
S ~ ~ 2~

means the composition of the values V0 and V~ to a
composite value, which is denoted by Ro.
The interpretation of the second example shows that
the assignment comprises the semantic meaning of an
(initialized) identity declaration for a variable: The
identifier R0 of the mode 2~ is used to denote the elab-
orated value on the left-hand side.
If in a program more than one assignment to the
same result or intermediate value variable occurs, then
the (dynamically) first assignment is to be interpreted
as an (initialized) identity declaration for a variable,
while all others are ordinary assignments. This would
give the genuine concept of a variable. On the other
hand, the initialization of an input parameter in con-
nection with a subroutine call,/° as well as the initializa-
tion of constants, can be interpreted to be an ordinary
identity declaration. However, these fine distinctions
are reflected neither in the notation nor in the explana-
tion of the semantics [Z59, p. 70]. Nevertheless, they
have strongly influenced Rutishauser's ideas, as seen
from ALGOL58.
The usual arithmetic and Boolean operations are
provided for, and they allow one to form expressions
(Ausdr~cke) in connective formula notation, n Besides,
comparison operations like = , ~ , < , with Boolean
values as results, can be used. For arithmetic opera-
tions, objects of the mode bits (denoted by S l . n ) are
interpreted as numbers in direct (lexicographic) corresponding

Further Operational Features
Apart from the possibility of selecting record and
array components by (component) subscripts, certain
operations from the predicate calculus are used to test
components with respect to a specified property, with
the result of selecting them or of obtaining a Boolean
value. In this respect, the Plankalk~2l surpasses the
potentialities in today's programming languages, in-
cluding ALGOL68.
Zuse uses both the "existence" and the "all" oper-
ator, and in particular the operator g:
gx(x ~ v A R(x))
0
means "The next component of V0, for which the
property R holds."


The property R, in the notation R(x), is expressed
by means of a computational rule which gives a Boolean
value (Ja-Nein-Wert), or of a result parameter of a
suitable subroutine (see Section 8).
It is clear, that procedures can be defined in, say,

ALGOL 68, which have the above effect. But it may be
worthwhile to see whether Zuse's constructions could
be introduced as original conce.pts in high level lan-
guages. See also [BG72].


Statements and Subroutine Calls
Statements are what Zuse calls Planteile. In partic-
ular, assignments are statements. Other statements,
which we shall discuss, are conditional statements and
repetitive statements. There is also a compound state-
ment, formed with the help of parentheses. In order to
separate statements, as well as the line marks (see Sec-
tion 5), a vertical bar is used.
Conditional statements are formed with the help of
the Bedingt-Zeichen -:-> (or .-~) in the following form
(B --~, (~,
where the condition (Bedingung) ~ is an expression with
Boolean value, and 0~ an arbitrary statement. The
elaboration of this conditional statement bedingter
Planteil) begins with ~ and ends with ~ or is continued
with a, depending on whether 6~ produces the value
0 = nein or L --- ja. An alternative for a in the first
case cannot be specified

The following example of a repetitive statement, that
is initiated by the letter W, shows an application of the
g-operation of the preceding section:
[ ~ u x ( ; ' V A x ' V ) ~ Z I ( R A R I 7 ( Z ) ) ~ ! ]
V 0 1 0 0 1 0
S m~ cr ~r 0 0

The elaboration of this Wiederholungsplan starts
with the first assignment. The left-hand side formula of
this assignment produces at each elaboration the next
component V0[i] which is different from V1, provided
it exists. In this case, the following statement is elabo-
rated and the process starts again. If, however, no com-
ponent is found then Z0 is unchanged and the elabora-
tion of the repetitive statement is finished.
In the second assignment of this example, where an
initialization of R0 is presupposed, R171(Z0) is the call
of a subroutine P17 (see Section 9), which is specified
to have one input parameter and a result parameter R1.

The elaboration of this call means the
identification of the actual parameter Z0 with the formal
input parameter, and following this, the elaboration of
Pl7. The value of the call is the value which is obtained
by R1.

It cannot be excluded that Zuse considered the input parame-
ters to be genuine variables whose values can be changed during
the subroutine. This is indicated by an isolated occurrence of
(V, V) ==~V in [Z59].
5 6 7
1~[Z49, p. 447]: "The Ergibt-Zeichen~joins an expression
which is to be calculated (left) with a result (right)." According to
Zuse such expressions mean computational rules (Rechenvor-
sehr~ften.)
~ The example in [Z59,p. 71] ends, however,with an expression g
instead of ~;~ R0 FIN, where R0 is the only result parameter